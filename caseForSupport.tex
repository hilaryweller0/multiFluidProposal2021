\documentclass[11pt,a4paper]{article}
\usepackage{amsmath,amssymb,amsfonts,url}
\usepackage{xcolor}
\usepackage[margin=2cm]{geometry}
\usepackage[T1]{fontenc}
\usepackage{ pifont}
\usepackage{multicol}
\usepackage{ulem}
\usepackage{todonotes}   %[disable] option
\allowdisplaybreaks

\renewcommand{\rmdefault}{phv}

\usepackage{pgfgantt}
\newcommand\textganttbar[4]{%
    \ganttbar{#1}{#3}{#4}
    \ganttbar[inline]{#2}{#3}{#4}
}

% ----------------------------------------------------------------
\vfuzz2pt % Don't report over-full v-boxes if over-edge is small
\hfuzz2pt % Don't report over-full h-boxes if over-edge is small

% Hilary's addition to make a compact article and use latex section commands
% make list and enumerate more compact
\usepackage{tweaklist}
\renewcommand{\itemhook}
{
    \setlength{\topsep}{3pt}
    \setlength{\parskip}{0pt}
    \setlength{\parsep}{0pt}
    \setlength{\partopsep}{0pt}
    \setlength{\itemsep}{0pt}
    \setlength{\labelwidth}{10pt}
    \setlength{\leftmargin}{\labelwidth}
}

\renewcommand{\enumhook}
{
    \setlength{\topsep}{3pt}
    \setlength{\parskip}{0pt}
    \setlength{\parsep}{0pt}
    \setlength{\partopsep}{0pt}
    \setlength{\itemsep}{3pt}
    \setlength{\labelwidth}{10pt}
    \setlength{\leftmargin}{\labelwidth}
}

\renewcommand{\deschook}
{
    \setlength{\topsep}{3pt}
    \setlength{\parskip}{0pt}
    \setlength{\parsep}{0pt}
    \setlength{\partopsep}{0pt}
    \setlength{\itemsep}{0pt}
    \setlength{\labelwidth}{0pt}
    \setlength{\leftmargin}{\labelwidth}
}

% Compact sections and parts
\makeatletter
\def\@part[#1]#2
{%
    \refstepcounter{part}%
    {%
        \parindent \z@ \raggedright \interlinepenalty \@M
        \normalfont \Large\bfseries\raggedright
        \partname\nobreakspace\thepart : \nobreakspace #2 %\markboth{}{}\par
    }%
    \nobreak \vskip 1.3ex \@afterheading%
}
\renewcommand\section
{%
    \@startsection {section}{1}{\z@}{-1ex \@plus -0.5ex \@minus -.1ex}%
   {0.5ex \@plus.1ex}{\large\bfseries\raggedright}%
}
\renewcommand\subsection%
{%
    \@startsection {subsection}{1}{\z@}{-1ex \@plus -0.5ex \@minus-.1ex}%
   {0.5ex \@plus .1ex}{\normalfont\bfseries\raggedright}%
}
\renewcommand\subsubsection%
{%
    \@startsection {subsubsection}{1}{\z@}{-0.5ex \@plus -1ex \@minus -.2ex}%
   {0.1ex \@plus .1ex}{\normalfont\bfseries\raggedright}%
}
\renewcommand\paragraph{\@startsection{paragraph}{4}{\z@}%
                                    {0.5ex \@plus0.5ex \@minus.1ex}%
                                    {-0.5em}%
                                    {\normalfont\normalsize\bfseries}}

% subsubsections are actually work packages
\renewcommand{\thesubsubsection}{WP\arabic{subsubsection}}% \hspace{-1em}}
\newcommand\workPackage{\@startsection{subsubsection}{3}{\z@}%
                       {-1ex \@plus -0.5ex \@minus -.1ex}%
                       {0.5ex \@plus .1ex}{\normalfont\bf\raggedright}}

\setcounter{secnumdepth}{5}

% paragraphs are sub work packages
\renewcommand{\theparagraph}{WP\arabic{subsubsection}\alph{paragraph}}%
\newcommand\subworkPackage{\@startsection{paragraph}{4}{\z@}%
                                    {0.5ex \@plus0.5ex \@minus.1ex}%
                                    {-0.5em}%
                                    {\normalfont\normalsize\bfseries}}

\def\@maketitle
{%
  \begin{center}%
  \let \footnote \thanks
    {\large\bf \@title \par}%
    \vskip 0.5em%
    {\normalfont
      \lineskip 1em%
      \begin{tabular}[t]{c}%
        \@author
      \end{tabular}\par}%
    \vskip 0.5em%
    {\large \@date}%
  \end{center}%
  \vskip -2.5em
  \par
  \vskip -1.5em
}
% Reduce the spacing around equations
\AtBeginDocument{%
 \abovedisplayskip=6pt plus 6pt minus 4pt
 \belowdisplayskip=6pt plus 6pt minus 3pt
 \abovedisplayshortskip=0pt plus 3pt
 \belowdisplayshortskip=7pt plus 3pt minus 4pt
}
\setlength{\jot}{0pt}% Inter-equation spacing
\makeatother

% Bibliography stuff
%\usepackage[square,sort&compress,numbers,super]{natbib}
\usepackage[round,sort&compress]{natbib}
%\usepackage[style=authoryear,citetracker,backref, backend=biber]{biblatex}

\setlength{\bibsep}{0pt}
\setlength{\bibhang}{6pt}

% modification to natbib to remove margin
\makeatletter
\renewcommand\NAT@bibsetnum[1]{\settowidth\labelwidth{\@biblabel{#1}}%
%   \setlength{\leftmargin}{\labelwidth}\addtolength{\leftmargin}{\labelsep}%
   \setlength{\leftmargin}{0pt}\addtolength{\leftmargin}{0pt}%
   \setlength{\itemsep}{\bibsep}\setlength{\parsep}{\z@}%
   \setlength{\itemindent}{\bibindent}%
   \ifNAT@openbib
     \addtolength{\leftmargin}{\bibindent}%
     \setlength{\itemindent}{-\bibindent}%
     \setlength{\listparindent}{\itemindent}%
     \setlength{\parsep}{0pt}%
   \fi
}
\makeatother

\usepackage[T1]{fontenc}
\usepackage{pgfgantt, pifont}
\usepackage{multicol}
\usepackage{todonotes,ulem}

\renewcommand{\rmdefault}{phv}

\begin{document}

\title{Case For Support \\ \Large
Multi-fluid parameterisation of convection
}
\author{Hilary Weller \and Georgios Efstathiou \and John Thuburn \and William McIntyre \and Daniel Shipley}
\date{}
\maketitle

\part{Track Record}

\paragraph*{Dr Hilary Weller (Reading)} has been the lead PI on three NERC grants on atmospheric modelling, a PI on both phases of the Met Office/NERC/STFC UK Dynamical Core project ``Gung-Ho'', to design and build the next Met Office dynamical core and a Co-I on both phases of the Met Office/NERC Paracon projects to improve the modelling of convection in atmospheric models.

Dr Weller has had successful collaborations with the Met Office staff, including creating and analysing long time step transport schemes \cite[]{CWPS17,SWMD17} and finding optimal coupling between fast and slow processes \cite[][]{WLW13}. She led pioneering work analysing numerical methods for quasi-uniform grids of the sphere \cite[e.g.][]{WWF09,Wel12,WTC12} and proposed improvements in modelling flow over orography \cite[]{WS14}. 

Before working on numerical methods, Dr Weller worked on tropical meteorology \cite[e.g.][]{LGWS09} which kindled her interest in atmospheric convection. She has combined her applied meteorology experience with her numerical modelling expertise, proposing, with Prof. Thuburn, the use of multi-fluid modelling for representing sub-grid-scale convection \cite[]{TWV+18} and creating the first numerical method to solve these equations \cite[]{WM19}. \cite{WMS20} demonstrates a skillful representation of sub-grid dry convection using a multi-fluid model.

\paragraph*{Dr Georgios Efstathiou (Exeter)} has over 15 years of experience in research on the modelling of
atmospheric processes at various scales, from turbulent motions in the boundary layer
to heavy precipitation synoptic systems. An overarching theme is understanding the
connections between atmospheric scales with the aim to improve high-resolution
numerical weather prediction. He has conducted many Large Eddy Simulation (LES) studies and used LES to identify 
the characteristics of the boundary-layer grey zone \citep[e.g.][]{efstathiou2015}
and develop parameterisations suitable for sub-kilometre, very high-resolution models 
\citep{efstathiou2016}. One of his main contributions in grey zone 
studies was the extension of dynamic sub-grid models from the LES to the grey zone region 
providing adaptive and scale dependent turbulence length scales for sub-grid models 
\citep{efstathiou2018,efstathiou2019a}. As part of the NERC/Met Office ParaCon 
project he has developed a novel method to identify updrafts in convective flows 
by optimising the multi-fluid decomposition of the atmosphere \citep{efstathiou2019b}.

\paragraph*{Prof. John Thuburn (Exeter)} holds a Chair in Geophysical Fluid Dynamics at the University of Exeter, jointly
funded by the Met Office under the Met Office Academic Partnership.
Since 2000 he has collaborated closely with the Met Office on numerical methods for their
weather and climate models. He made important contributions to the development of the
ENDGame dynamical core \cite[e.g.][]{WSW+14}, which is now a major operational success.
% Could drop the next two sentences
Since 2011 he has collaborated with the Met Office and other UK academic partners on the ``Gung-Ho''
project to develop a future dynamical core suitable for massively parallel computer
architectures. An important theme is to capture key aspects of accuracy related to balance and
conservation on non-traditional grids \cite[e.g.][]{TC15}.

The coupling of physical parameterisations and subgrid models to resolved dynamics is often
particularly subtle because the coupling occurs via marginally resolved and imperfectly
represented scales. Prof.\ Thuburn has contributed to understanding these numerical aspects
of physics-dynamics coupling in the context of quasi-two-dimensional turbulent cascades \cite[e.g.][]{TKW14} and boundary layer parameterisations \cite[]{HTW13a}.
Most relevant for the present proposal is that,
together with co-authors, he developed the mathematical framework for the multi-fluid
approach \citep[][]{TWV+18}, analysed the conservation and normal mode properties
of the unparameterised multi-fluid equations \citep[][]{TV18}, and demonstrated
a proof of concept for a single-column model of the dry convective boundary-layer \citep[][]{TEB19}.


\section*{Institutions}

\paragraph*{The University of Reading Meteorology department} is one of the largest of its kind in Europe with 50 academic staff, 20 senior research staff and fellowship holders, around 90 postdocs and around 70 PhD students. In the 2014 Research Excellence Framework (REF), 86\% of their research was graded as world leading or internationally excellent. Their ``research power'' places them 3rd in the country in Earth Systems and Environmental Science, and the impact of their research was rated 9th highest in the country. The University is a formal Academic Partner of the Met Office (MO) and hosts about 20 Met Office scientists. The Reading PDRA will attend Mesoscale group meetings and weekly seminars on atmosphere and ocean science.

\paragraph*{The Department of Mathematics at the University of Exeter} includes the Geophysical and Astrophysical Fluid Dynamics and Exeter Climate Systems research groups, who between them have over~50 staff and over~40 PhD students researching topics related to weather, climate, and modelling. Both groups have excellent track records of collaboration with the Met Office. In the 2014 REF, for UoA 10 (Mathematical Sciences) 83\% of their research was assessed as world leading (4*) or internationally excellent (3*), while for UoA~7 (Earth Systems and Environmental Sciences) 89\% of their research was assessed as 4* or 3*. The Exeter PDRA will \todo[inline]{attend something?}

\paragraph*{The Met Office} is a world leader in weather forecasting and climate prediction, and their atmospheric model is used by many operational centres (for example, the Australian Bureau of Meteorology). The convection parameterisation group, led by Dr Alison Stirling, publishes widely about their research on aspects of on convection, weather prediction and parameterisation. They have just developed a new flexible convection scheme called CoMorph, which will be the basis for future convection developments at the Met Office. Alison's group contributes to ensuring that the Met Office model maintains its high skill and the Met Office maintains its international reputation in numerical weather prediction, climate projection and research. 

\input{trackRecord.bbl}

\newpage

\part{Research Proposal}

\section{Motivation and Summary}

The representation of sub-grid-scale convection is arguably the weakest aspect of weather and climate models, leading to unrealistic simulations of monsoons, the diurnal cycle of rainfall, the Madden-Julian Oscillation and convectively coupled waves  \cite[]{SAB+13,HPB+14}. The representation of the sub-grid-scale flow becomes particularly challenging at resolutions where convection is partially resolved -- the grey zone -- as assumptions behind current schemes such as horizontal homogeneity are grossly violated \cite[eg.][]{GG05}.

Standard convection schemes are based on a crude representation of vertical fluxes in terms of a division of the sub-grid flow into separate, homogeneous flows such as updraft, downdraft and environment, with assumptions of small updraft area fraction and horizontally homogeneous quasi-equilbrium, implying no net mass flux and no horizontal transports. The multi-fluid approach removes these assumptions and has shown promise in single column models but is as yet unproven in three dimensional models and does not address sub-grid variability beyond the multi-fluid division. High-order turbulence modelling addresses the moments of probability distributions of sub-grid variability. The multi-fluid and high-order turbulence (multi-moment) approaches thus provide two different ways of accounting for unresolved variability in models, each likely to be most useful in different regimes. Their unification has the potential to work well across a much wider range of regimes, in particular, enabling the construction of a scale aware model that is applicable at a range of resolutions encompassing the turbulent and convective grey zones.

A three-dimensional multi-fluid, multi-moment model of convection with a seamless transition from fully resolved to grey zone to fully parameterised convection will be developed by combining existing models. This has already been shown to work well in a single-column so we will start by simulating fully parameterised convection in three dimensions using the same closures for pressure fluctuations and for convective entrainment and detrainment and then move to partially resolved, grey-zone convection. High resolution Large Eddy Simulation (LES) will be used to diagnose multi-fluid and multi-moment budgets and evaluate closures. The new approach will be evaluated using a suite of equilibrium and non-equilibrium test cases of convective growth and propagation. A key deliverable will be an open source, community model enabling the future research needed to develop the approach towards operational use.

\section{Scientific Background}

At coarse resolution, convection parameterisation is necessary because, without it, convection is forced to occur at the model grid scale resulting in highly unrealistic behaviour and instability \cite[]{PY15}. Schemes such as mass flux convection remove this instability by re-distributing heat, moisture and momentum in the vertical but not mass \cite[]{Tied89,GR90}. Convection is assumed to be in equilibrium with the large scale so no account is taken of the gradual build-up of convection. Convection in one grid column is assumed to be independent of its neighbours and advection of convective systems is neglected. There have been valuable improvements such as relaxing the quasi-equilibrium assumption \cite[]{PR98,GG05,Par14} and stochasticity \cite[]{PC08} but due to the fundamental assumptions in convection parameterisation it has not been possible to create schemes that work well when convection is partially resolved -- the grey zone.

\cite{TWV+18} proposed a new approach: multi-fluid modelling of convection, in which convective updrafts, downdrafts and the stable environment are modelled as distinct fluids with separate, but consistent, momentum, continuity, energy and moisture transport equations. The fluids interact via entrainment, detrainment and pressure terms. Consequently non-equilibrium, net mass transport and other horizontal and vertical transports are handled by these equations rather than by empirical parameterisation. This is related to the extended EDMF scheme of \cite{TKP+18}. Multi-fluid modelling needs some of the same closures as mass-flux parameterisations such as entrainment and detrainment. Convection parameterisations also include trigger functions and cloud base mass flux (which in this proposal are described as a form of entrainment). There have been other attempts to include net mass transport by convection \cite[]{KB08,MB19} by modifying existing models in a way which is unlikely to work in the grey zone since they do not have numerically stable coupling between the updraft velocity and the pressure.

The idea of two fluid partitions, each with simple univariate or bivariate Gaussian distributions of sub-grid variability was used by \cite{GLC02} and forms the basis of the CLUBB parameterisation. However, specifying the parameters in such distributions requires knowledge of (or assumptions about) many high-order moments; multi-fluids each with first or second-order moment equations provides potentially a simpler but more powerful approach.

Much work on parameterising turbulence in the atmosphere has focused on the boundary layer. Most relevant to this project is the \citet{mellor1973,mellor1974,mellor1982} hierarchy of turbulence closure models. The continuous governing equations were ensemble averaged and manipulated to obtain prognostic equations for all second-order moments such as turbulent kinetic energy (TKE) and sub-grid-scale heat fluxes. For this to work as part of a parameterisation of deep convection, prognostic equations for second-order moments and fluxes within each fluid are needed which will need additional closures.

As supercomputers increase in size, convection is better resolved over larger areas \cite[eg.][]{GC17}. Computers will never be big enough to fully resolve convection over the whole globe in weather predictions \cite[eg.][]{SSJ+19} so parameterisation will always be needed. In order to make sense of terabytes of convection resolving simulations, machine learning has been used to find relationships between behaviour that is sub-grid-scale with respect to a coarse model and variables that are resolved by the coarse model. Machine learning with neural networks may be able to discover processes from high resolution data that have not been observed \cite[eg.][]{ogorman2018}. However, due to their lack of physical basis, neural networks may not be able to extrapolate to new situations such as global warming. An alternative which enables finding closed-form parameterisations is the relevance vector machine \cite[eg][]{tipping2001}. This uses a sparse Bayesian regression to discover which combination of resolved variables best reproduces a desired subgrid term diagnosed from a high-resolution simulation. This results in a directly interpretable parameterisationand has been used to suggest a successful scale-aware parameterisation for ocean mesoscale eddies \cite[]{zanna2020}.

\section{More detail on Multi-fluid Modelling of Convection}
\label{sec:mf}

The multi-fluid, dry Boussinesq Navier-Stokes equations \cite[approximated by][]{WMS20} are:
\begin{eqnarray}
\frac{\partial\sigma_{i}}{\partial t}+\nabla\cdot(\sigma_{i}\mathbf{u}_{i}) & = & {\textstyle\sum}_{j\ne i}\left\{ M_{ji}-M{}_{ij}\right\} \label{eq:sigma}\\
\frac{D_{i}\mathbf{u}_{i}}{Dt}+\nabla P_{i} & = & b_{i}\mathbf{k}+\nabla\cdot\left( \nu_i\nabla\mathbf{u}_{i}\right)+\frac{1}{\sigma_{i}}{\textstyle\sum}_{j\ne i}\left\{ M_{ji}\left(\mathbf{u}_{ji}^{T}-\mathbf{u}_{i}\right)-M_{ij}\left(\mathbf{u}_{ij}^{T}-\mathbf{u}_{i}\right)\right\} \label{eq:mom}\\
\frac{D_{i}b_{i}}{Dt} & = & \nabla\cdot \left(\alpha_i \nabla b_{i}\right)+\frac{1}{\sigma_{i}}{\textstyle\sum}_{j\ne i}\left\{ M_{ji}\left(b_{ji}^{T}-b_{i}\right)-M_{ij}\left(b_{ij}^{T}-b_{i}\right)\right\} \label{eq:b}\\
{\textstyle\sum}_{i}\nabla\cdot\sigma_{i}\mathbf{u}_{i} & = & 0\label{eq:divFree}\\
{\textstyle\sum}_{i}\sigma_{i} & = & 1.\label{eq:sumOne}
\end{eqnarray}
where $\sigma_i$, $\mathbf{u}_i$, $b_i$ and $P_i$ are the volume fraction, velocity, buoyancy and pressure of fluid $i$. Representing moist convection also needs transport equations for moisture in each fluid and a latent heating term in the buoyancy equation. $M_{ij}$ is the mass transfer rate from fluid $i$ to $j$ which is equivalent to entrainment, detrainment and cloud base mass flux. $\mathbf{u}_{ij}^T$ is the mean velocity of the fluid that is transferred from $i$ to $j$ and $b_{ij}^T$ is the mean buoyancy of the fluid that is transferred. The diffusion terms $\nabla\cdot\left( \nu_i\nabla\mathbf{u}_{i}\right)$ and $\nabla\cdot \left(\alpha_i \nabla b_{i}\right)$ are approximations of sub-grid-scale fluxes which occur because each fluid is not uniform at sub-grid-scales, with $\nu_i$ and $\alpha_i$ being effective turbulent diffusivities.

These equations can represent sub-grid-scale convection if, for example, fluid 0 is the stable environment, fluid 1 is buoyant convection and fluid 2 is downdrafts. In order to initialise a multi-fluid model, high resolution data must be divided into separate fluids and then averaged onto the (coarser) model grid. This is conditional averaging. The same approach is needed to evaluate the multi-fluid model and to diagnose sub-grid-scale fluxes from high resolution data. Techniques such as the optimisation of resolved heat fluxes \cite[]{efstathiou2019b} can be used for conditionally averaging.

For the multi-fluid equations to represent sub-grid-scale convection, further closures are needed:\hspace{-1in}
\begin{itemize}
\item Mass transfers between fluids, $M_{ij}$, to represent entrainment and detrainment.
\item The mean buoyancy, moisture and momentum of the fluid that is transferred, $b_{ij}^T$, $q_{ij}^T$ and $\mathbf{u}_{ij}^T$. 
\item Pressure differences between fluids.
\item Turbulent diffusivities of buoyancy, moisture and momentum to represent sub-grid-scale fluxes for each fluid and their interactions. These can be parameterised directly or predicted by the prognostic multi-moment equations which require more closures.
\end{itemize}
Closures such as entrainment and detrainment rates and turbulent diffusivities from conventional parameterisations can be used with the multi-fluid equations. Complete closure sets specific to multi-fluid modelling have been developed \cite[]{WMS20,TEB19} and proved successful in single-column modelling. We therefore have good starting points for closing the multi-fluid equations but none have yet been shown to work in three dimensions or in the grey zone. 

\section{Relationship with ParaCon}

The poor state of convection parameterisation was the motivation for the \pounds 10M joint NERC, Met Office ParaCon project which aims to make a ``step change in our ability to predict weather and climate impacts'' by improving parameterisation suitable for existing dynamical cores. This project will use some ParaCon outcomes:
\begin{enumerate}
\item The flexible convection parameterisation scheme CoMorph has been developed for use with the Met Office Model. This includes new closures for entrainment and detrainment.

\item The MONC LES model was developed and used to create simulations of convection archetypes. Simulations of the dry and moist convective boundary layer, BOMEX shallow cumuli over the tropical ocean \cite[]{HR73}, radiative-convective equilibrium (RCE), diurnal cycles of deep convection, the shallow cumulus ARM case \cite[]{BCC+02} and the transition from shallow to deep LBA case \cite[]{BFGB02} are available at multiple resolutions.

\item High resolution simulations of two dimensional Rayleigh-B\'enard convection are available for a range of Rayleigh numbers.

\item A method of partitioning the atmosphere into coherent thermal structures and their environment was developed \cite[]{ETB20} by maximising the heat that is transported by the mean velocity of each fluid.

\item A moist, two-fluid, single column model was developed using a variety of entrainment/detrainment closures dependent on turbulent kinetic energy, vertical velocity convergence and atmospheric instability. This is able to reproduce the updraft area fraction ($\sigma_1$) and cloud properties of the ARM case \cite[]{BCC+02} (fig \ref{fig:clouds}). 

\begin{figure}
\begin{tabular}{cc}
	\includegraphics[width=0.24\linewidth]{fromWill/04_Proposal_Exeter_Progress/updraftFraction_32.png}&
	\includegraphics[width=0.74\linewidth]{fromWill/04_Proposal_Exeter_Progress/timeseries_cloud_height2.png}\\
	$\sigma_1$ & Time (hours)
\end{tabular}
	\caption{Left: the updraft fraction from the large eddy simulation (solid) and the two-fluid single-column model (dashed) at nine hours.
	Right: The boundary layer height (blue), cloud base (black) and cloud top (red) for the ARM test case. Large eddy simulation (solid) and two-fluid (dashed).}
	\label{fig:clouds}
\end{figure}

\item A multi-fluid analogue of the Mellor-Yamada hierarchy was derived including second-moment equations for each fluid type. All of the terms in the original Mellor-Yamada formulation have analogues in the multi-fluid version. There are new terms accounting for relabelling of fluid types (i.e.\ entrainment and detrainment) and  terms that account for sub-grid-scale pressure fluctuations.

\item Fully compressible multi-fluid models with simplified moist physics were developed. MultiFluidAtmosFOAM is a three-dimensional model written using OpenFOAM enabling focus on the equations and algorithms rather than the spatial discretisation or parallelisation.

\end{enumerate}

The multi-fluid approach is a technique to model convection across a range of scales rather than a stand-alone parameterisation that can be called from an existing dynamical core; it requires the bottom-up re-development of the model formulation. For this reason, the development of the multi-fluid approach was always expected to extend beyond the end of ParaCon. In this project we will build on the progress made under ParaCon in developing both the multi-fluid approach (basic formulation, stable numerical solutions, sub-grid closures, and single column demonstrations) and the multi-moment turbulence approach by bringing them together.

\section{Proposed Research}

\subsection{Objectives}

The over-arching aim is to create a three-dimensional multi-fluid model of the atmosphere which is accurate across a range of convection resolving to convection parameterising resolutions with three essential, minimum properties:

\begin{enumerate}\renewcommand{\theenumi}{\alph{enumi}}
\item When convection is well resolved, the solutions are identical to a single fluid model.
\item When convection is fully parameterised, the solutions are at least as accurate as a dynamical core with a conventional parameterisation.
\item When convection is partially resolved, the solutions are more accurate than a single-fluid model with no parameterisation and more accurate than the multi-fluid model at coarse resolution.
\end{enumerate}

We plan to meet these aims via these specific objectives:

\begin{enumerate}
\item\label{it:model} Develop a flexible, three-dimensional multi-fluid, multi-moment model of the atmosphere including two phases of moisture. This objective does not include finding optimal closures.

\item\label{it:budgets} Calculate budgets of sub-grid-scale fluxes from LES data in order to diagnose the values of entrainment and detrainment and other closures.

\item Find and test suitable closures so that the single-moment, multi-fluid model represents under-resolved {\it dry} convection accurately using the the techniques described in section \ref{sec:tools}.

\item Find and test suitable closures so that the single-moment, multi-fluid model represents under-resolved {\it moist} convection accurately using the the techniques described in section \ref{sec:tools}. 

\item Combine multi-fluid and Mellor-Yamada modelling to develop a  multi-fluid, multi-moment model for turbulent convection including consistent interactions between moments and fluids and closures for unknown correlations between moments.

\item Evaluate using a hierarchy of convective cases from the Paracon project.

\item Deliver an open access, well documented, parallel, multi-fluid model of the atmosphere with well documented performance at representing convection across scales.
\end{enumerate}

\subsection{Work Packages}
\label{sec:WPs}

\workPackage{Flexible Modelling Framework (Reading, 6 months) \label{WP:model0}}

Using the OpenFOAM flexible modelling framework and using the code developed during Paracon we will develop multiFluidAtmosFOAM, an anelastic, three-dimensional multi-fluid, multi-moment including moisture and some simple transfer terms and closures. This work will involve software testing and testing of exact properties such as conservation but will not aim to represent convection accurately. Combining code from Reading and Exeter will enable us to benchmark against the existing codes. All code will be version controlled and uploaded to a public \url{github.com} repository. A model description paper will be submitted to Geoscientific Model Development. This model will be suitable for carrying out the research in the rest of the project. 

Some of the multi-fluid code developed during Paracon was fully compressible and some was Boussinesq. We do not need to include the complexities of full compressibility in multiFluidAtmosFOAM as this is not central to convection and the LES model that we are comparing with is anelastic. 

\workPackage{LES Budgets (Exeter, 9 months) \label{WP:LESbudgets}}

We will carry out detailed diagnostics of the LES data created during ParaCon, computing complete budgets for all second-moment quantities for the cases of a single fluid, two fluids (updraft and environment), and, for LBA, three fluids (updraft, downdraft, and environment).
We will decompose the flow into updrafts and their environment using the optimal decomposition developed during Paracon and extend this to diagnose coherent downdrafts. The budgets will be calculated for a range of horizontal filter scales, from a global horizontal average down to scales comparable to the cloud width and boundary layer depth. From the LES we can calculate the size but not the form of the closures.

\workPackage{Multi-fluid modelling of dry convection (Reading, 6 months) \label{WP:dryMF}}

An important aspect of our approach is to find closures within the simplest possible framework to avoid multiple competing influences contaminating the results. Hence we will ensure that the entrainment and detrainment and pressure differences between fluids are correctly modelled for dry convection first using the tools described in section \ref{sec:tools} and using Paracon data.

An important aspect of multi-fluid modelling is that it enables net mass transport by convection. Rayleigh-B\'enard convection is horizontally homogeneous at large  scales so there is not horizontal mass transport. However when the convection is partially resolved, the parameterised part of the flow will need to transport mass in the same way as the resolved flow. Important aspects of multi-fluid modelling can therefore be tested in this simple setting.

\workPackage{Multi-fluid model of moist convection (Reading, 9 months) \label{WP:moistMF}}

In keeping with our strategy to keep the modelling as simple as possible while capturing important aspects of convection, the next step is to add moisture to the Boussinesq model and simulate Rainy-B\'enard convection \cite[]{WPS10,VPT19} which has known solutions and provides a link between Rayleigh-B\'enard convection and atmospheric convection. Rainy-B\'enard convection will be simulated at high resolution with a single fluid model to provide a reference for coarse resolution multi-fluid model. The results in \cite{VPT19} imply that the flow can be separated into saturated updrafts and unsaturated turbulent regions providing a simple criterion for conditional averaging the high resolution results. Hence entrainment and detrainment rates and pressure differences between fluids can be diagnosed from the reference simulation and additional closures in comparison to the dry case can be found. Closures will be found that work through the grey zone with the limiting cases of fully parameterised and fully resolved convection.

Once closures have been developed for simplified microphysics in the Rainy-B\'enard setting, a more general model will be developed in order to tackle more complex cases from the hierarchy, feeding into the evaluation (\ref{WP:evaluate}).

\workPackage{Multi-moment, multi-fluid modelling of dry convection (Exeter, 9 months) \label{WP:YMdry}}

The different levels of closure model proposed by Mellor and Yamada are based on neglecting transience, advection, and third-order turbulent transport in various second-moment equations. Using the budgets calculated in \ref{WP:LESbudgets}, we will test the validity of making analogous approximations for different numbers of fluids and for different filter scales, both in the boundary layer and in the cloud layer. The diagnosed distributions of third-order terms, pressure-correlation terms, and dissipation terms will be compared with the models proposed by Mellor and Yamada (and subsequent authors). The length scales required to optimise the multi-fluid Mellor-Yamada scheme will be compared for different filter scales and different numbers of fluids, paying particular attention to the grey zone. 

\workPackage{Multi-moment, multi-fluid modelling of moist convection (Exeter, 12 months) \label{WP:YMmoist}}

Building on the LES simulations and the single-column, multi-fluid, multi-moment simulation of the ARM test case (see fig \ref{fig:clouds}), we will simulate this test case with the three-dimensional multi-fluid, multi-moment model at coarse resolution with the same closures as the single-column model. Due to the horizontal homogeneity of reference LES data, the three-dimensional model should behave the same as the single-column model. We will next increase the resolution into the grey zone where existing parameterisations may not work so well. We will explore if additional moments or a third fluid are needed in order to best represent the sinking of overshooting thermals. 

We will simulate radiative-convective equilibrium and compare with Paracon LES data. We will check that the equilibrium conditions are not sensitive to modelling assumptions about the initial conditions and are not resolution dependent through the grey zone.


\workPackage{Evaluation \label{WP:evaluate} (Exeter, 6 months, Reading 9, months)}

The evaluation will use different test cases with different forcings in order to test if the new model is predictive rather than a fit to LES. This will consist of a squall line over flat terrain \cite[]{FM06}. In order to test if we are doing better than a state of the art conventional parameterisation, we will compare with the idealised Met Office UM model using the CoMorph parameterisation at coarse resolution using the same simplified microphysics. Squall lines are known to be challenging for standard convection parameterisations \cite[e.g.][]{LCD+08} due to the lack of propagation of convection and the lack of mass transport by convection. We will evaluate sensitivity to resolution through the grey zone. 

We will compare different versions of multiFluidAtmosFOAM with different closures, different numbers of fluids and different moments in order to find how many terms and how many prognostic equations need to be retained. 

\workPackage{A community turbulent multi-fluids model \label{WP:model} (Reading, 6 months)}

MultiFluidAtmosFOAM will be developed throughout the project. The final anelastic version will:
\begin{enumerate}
\item solve for one, two or three fluids;
\item include prognostic equations for second moment of variables within each fluid;
\item use Cartesian geometry and no orography;
\item include simplified radiation and moist physics, a simplified bottom boundary and no stratosphere;
\item be parallelised using MPI.
\end{enumerate}
We will create documentation and a set of test cases for new users to run. A part II model description paper will be submitted to Geoscientific Model Development.

\subsection{Tools for Finding and Evaluating Closures}
\label{sec:tools}

Good candidates exist for closing the multi-fluid, multi-moment equations. The success of this project does not rely on finding new closures since we will start by evaluating existing closures. We have a toolkit for evaluating closures and finding new ones. We will pay particular attention to how relationships between sub-grid terms and resolved variables vary across scales. 

\subsubsection*{Parameter Correlations}

From the LES data we will calculate the parameters that make existing closures correct. If  parameters are constant or have a clear relationship with resolved variables then the closure may prove useful.

\subsubsection*{Assumed Probability Distribution Functions (PDFs)}

This approach was explored in the PhD thesis \cite{McIn20} supervised by Weller and can be used to find transfer terms. Fluid properties can be represented as bi-or tri-Gaussians using the mean and variance of each fluid. From these it is possible to calculate the mass of fluid that crosses a threshold that defines the fluids. This implies a mass transfer (similar to transferring the tails of the distribution). The mean properties of the fluid transferred can also be calculated from the distribution. 

\subsubsection*{Theory of Distributions}

Derivations of multi-fluid equations make use of discontinuous functions to label different regions of the underlying fluid \cite[]{Dopa77,TWV+18}. The terms requiring closure in such an approach depend on derivatives of those discontinuous functions. The theory of distributions (or generalised functions) \cite[]{Schw08} allows for the definition and manipulation of such discontinuous functions and their derivatives. This can be used to derive exact integral expressions for the terms requiring closure. Once conditions are specified to split the flow into multiple fluids, these conditions can be used to compute the closures directly from LES data. We will also use this method to suggest closures by considering simplified flows, and asymptotic approximations of evolution equations. For instance, the integrals can be calculated analytically for the first normal mode of Rayleigh-B\'{e}nard convection  \cite[]{SWCM2x}. This will lead to new physically-based closures for the entrainment and detrainment terms in a multi-fluid model. 

\subsubsection*{Asymptotics}

Candidate closures will be used to simulate various idealised test cases and we will demand:
\begin{enumerate}
\item Term-by-term Galilean invariance.
\item Two initially identical fluids should remain identical when mixed.
\item For any linear term of the multi-fluid equations, summing over all fluids should lead to the equivalent term of the original single-fluid equations.
\item  When convection is fully resolved, two fluids should evolve as one.
\item\label{it:energyTransfer} Closures and numerical methods should not create energy and not create or destroy first moments of primitive variables.
\item\label{it:boundedTransfer} Closures and numerical methods should not create unbounded values except where second-order moments provide sources for first-order moments.  Numerical methods to ensure \ref{it:energyTransfer} and \ref{it:boundedTransfer} were derived by \cite{MWH20}.
\item At coarse resolution, a two-fluid model should be more accurate than a single-fluid model at the same resolution. By more accurate we mean that conditional averages are closer to a fully resovled single-fluid model.
\item An initially empty fluid should not influence a non-empty fluid.
\end{enumerate}
Existing convection parameterisations do not have these properties which can lead to spurious solutions in the grey zone, for example parameterised convection leading to hot columns which violate the weak temperature gradient assumption.

Once a set of closures obeys the above constraints, we will evaluate by developing coarse resolution multi-fluid models of LES convection cases developed during Paracon. These will increase in complexity in order to maintain an unambiguous link between additional complexity and model fidelity. They will start as single column, then fully parameterised coarse resolution and finally partially resolved (grey zone). 

\subsubsection*{The Relevance Vector Machine (RVM)}

From the analysis of the LES (\ref{WP:LESbudgets} below) we will be able to calculate exactly the size but not the form of the closure needed to produce correct multi-fluid results. To apply the RVM we will also create a library of resolved variables which the specified subgrid term is conjectured to depend on. The RVM can then be used to discover which combination of resolved variables best reproduces the subgrid terms. This would provide physical insight into how the coherent structures of convection act, and to what extent that interaction can be universally represented.

\section{Potentials for Impact}

If we are able to show that multiFluidAtmosFOAM can represent convection more accurately than single-fluid models for a similar computational cost, then multiFluidAtmosFOAM can be further developed to create a more complete model that can be used for weather and climate prediction.

Earlier opportunities for impact will come from our data analysis. The closures that we develop for the multi-fluid equations are likely to be useful for other parameterisations of convection. Collaboration with the Met Office will ensure that useful developments see early operational use. 

\section{Management and Collaboration}

We plan to work as one team across both institutions and will meet together online once a week. Meetings with Met Office will be held four times a year to discuss project progress and plans and to co-ordinate evaluation, comparing with the idealised UM (see letter of support from the Met Office). Insights gained during this project will feed into the Met Office model development.

\section{Risks and Mitigations}

Some of the risks associated with this proposal are generic to modelling: slow code development and uncertainties in the results. These risks are mitigated by the experience of the investigators and researchers and by the flexible and advanced code that has already been developed.

There are some dependencies between work packages; the development and evaluation of closures relies on the LES analysis. However the theoretical work and the single-moment, multi-fluid model development can proceed without the results of the LES analysis. The evaluation relies on the model development work packages.

Multi-fluid modelling cannot be retro-fitted to an existing model as is usually done with convection parameterisations. This means that a whole new model is needed in order to produce a multi-fluid model of convection. This is a big undertaking for operational centres so the biggest risk is that multi-fluid modelling will not be adopted by big modelling centres. We therefore plan to create models with enough functionality to be useful for independent research into convection and parameterisation.

\renewcommand\refname{Reference (not included in Track Record)}
\input{proposedResearch.bbl}
%\newpage
%\renewcommand\refname{All References (not for submitted copy)}
%\bibliography{Weller,Thuburn,Shipley}
\bibliographystyle{myNat}

\end{document}
